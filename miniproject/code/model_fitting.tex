\documentclass[11pt]{article}
\usepackage{graphicx, caption, lineno, natbib}

\linespread{1.25}

\newcommand{\reporttitle}{Time Lag Is a Worthwhile Parameter in Mechanistic Modelling of Bacterial Growth}

\newcommand{\reportauthor}{Rui Zhang}

\newcommand{\wordcount}{2448}

\newcommand{\supervisor}{Samraat Pawar}

\date{1/21/2021}

\begin{document}
  \input{titlepage}
  
  \linenumbers
  
  \begin{abstract}
    There are a lot of different models fitting population growth which can be applied to the analysis of 
    ecosystem dynamics and emergent functional characteristics. This project focuses on cubic polynomial model, 
    Logistic model and Gompertz model,  trying to find out which model fits real microbes population growth 
    data the best. After using R to fit models and calculating Akaike information criterion, Schwarz criterion and 
    the residual sum of squares, Gompertz model shows the best fitting performance among these 3 models. The 
    parameter time lag is worthy to being added into microbial growth model considering both fit goodness and 
    model complexity.
  \end{abstract}
  
  \section{Introduction}
    Population growth equation is necessary in dynamic models to describe the development of the ecosystem 
    and interactions between the population and environment \citep{GAMITO199883}. 
    Carrying capacity, population growth rate and metabolic rate of population can all be affected 
    by environment and then results in changes of population dynamics and ecosystem dynamics \citep{BernhardtJoeyR2018MTat}. 
    Totally comprehension of population growth model is necessary to estimate extended domains, such 
    as marine fisheries productivity \citep{SARKER20171}, disease transmission \citep{AldilaDipo2019APDM}, 
    even the direction of evolution \citep{AndrewP.Hendry2017EDiC}.\\
    \\
    There are also many different population growth models, involving phenomenological and mechanistic models. 
    Cubic polynomial model is the simplest phenomenological mathematical linear model, with 4 parameters, to 
    describe population growth. For mechanistic models, a classical, somewhat mechanistic model is the 
    logistic equation with 3 parameters \citep{Peleg1997}. However this model failed to match lag phase of population growth. 
    To capture the lag phase, more complicated bacterial growth models have been designed, one of which
    is the modified Gompertz model with 4 parameters \citep{M.H.Zwietering1990MotB}, which is the most 
    frequently used model in the research to fit population growth. When comparing these models, 
    a 4 parameters totally phenomenological model, a 3 parameter mechanistic model and a 4 parameters 
    mechanistic model, Zwietering used t test and F test to compare models' statistical sufficient to describe 
    population growth and their ease of use \citep{M.H.Zwietering1990MotB}.\\
    \\
    Now there are some more scientific model selection criterion to answer the question whether it is 
    worthwhile to add a new parameter to improve the precise of model fitting. The residual sum of squares, 
    aka $R^2$, is a naive method to measure fit. But only considering fit is not enough. The Akaike 
    information criterion (AIC) \citep{BurnhamKennethP2002Msam} calculates the Kullback-Leibler information lost 
    during approximation full reality with the fitted model, which contains a bias correction factor with 
    respect to model complexity. Schwarz criterion, also named Bayesian information criterion (BIC), is 
    another model selection criterion considering both fit and complexity. However, it includes a penalty 
    term dependent on dataset size and tends to choose simpler models \citep{JOHNSON2004101}. Generally, AIC 
    is prefered by researchers because of based on Kullback-Leibler information.\\
    \\
    Usually, people favor simpler models due to Ockham's razor (aka parsimony principle) if models describe 
    the same pattern adequately \citep{Peleg2011}. To find out whether time lag is a worthwhile parameter to be 
    introduced in mechanistic model fitting and whether a more complicated form of equation works better 
    than totally phenomenological linear model, I fit cubic polynomial, Logistic and Gompertz models to 
    a large amount of microbes population growth experiments data with R. Bacteria cultivation is easy and 
    standardized, and it is convenient and fast to get population growth data. Furthermore, bacteria appears 
    to have a large biodiversity and large habitat diversity which plays an important role in human health, 
    food safty, fermentation engineering, environment and so on. R is a platform that is mature and convenient 
    to fit ecological model and analyze data, based on numerous support packages, help documents, simple 
    installation and convenient high-level syntax \citep{Benjamin2013}. By means of comparing statistical 
    measure of model fitting, we can discover which model is the most suitable to describe microbes population 
    growth, especially bacterial growth.

  
  \section{Materials \& Methods}
  \subsection{Materials}
    The dataset contains 4388 records, collected from 10 lab experiments data across the world, containing 
    the change in biomass or population of microbes over time. The species in these microbes 
    data include bacteria and mobile phytoplankton, which play an important role in food safty and 
    environment. Time range of the dataset is also quite large, varying from 1962 to 2018, which could 
    guarantee the universality of model comparison conclusions. 

  \subsection{Methods}
    \subsubsection{Computing tools}
      Data processing, model fitting and results visualization are performed with R due to its excellent 
      graphing capability and numerous packages for statistical analysis and data handling, including reshape2, 
      minpack.lm and ggplot2 packages used in this program. Python is an ideal tool to build an automated 
      workflow to analyze data with the subprocess module, especially useful when the program involves many 
      different languages. Therefore, I use shell commands in Python script to run R scripts and compile LATEX 
      to create the report, which guarantees this program is fully reproducible.


    \subsubsection{Data management}
      In this program, two main variables of interest are time and population. Therefore, I create 
      a new ID with all of the other variables, with which I divide the total dataset into 285 subdatasets. 
      Then I delete negative time and population biomass, during which I find an unreasonable dataset. This 
      subdataset contains 29 records but 19 biomass values are negative, so I delete this dataset totally. 
      Finally there are 284 datasets, containing 4284 records to fit cubic polynomial model, Logistic model 
      and Gompertz model.

    \subsubsection{Model fitting}
      There I choose cubic polynomial model, Logistic model and Gompertz model to fit all of 284 datasets. 
      Because of subsequent AIC and BIC comparison, the formation of biomass or population should be the same. 
      To guarantee all of these equation have the same response variable, I transform cubic polynomial model and 
      Logistic model into log scale.\\ 
      The equation of cubic polynomial model is:
      \begin{equation}
        \ log(N_{t}) = log(a + b t + c t^{2} + d t^{3})\
      \end{equation}
      The equation of Logistic model is:
      \begin{equation}
        \ log(N_{t}) = log(\frac{N_{0} K e^{r t}}{K + N_{0} (e^{rt} - 1)})\
      \end{equation}
      The equation of Gompertz model is:
      \begin{equation}
        \ log(N_{t}) = N_{0} + (N_{max} - N_{0})e^{-e^{r_{max} e \frac{t_{lag}^{-t}}{(N_{max} - N_{0}) log(10)}+1}}\
      \end{equation}
      a, b, c, d are all parameters without biological meaning. 
      $N_{t}$ is biomass or population value at time t, $N_{0}$ is biomass or population value at initial condition, 
      $N_{max}$ and K are carrying capacity, r and $r_{max}$ are maximum growth rate and $t_{lag}$ is time lag which 
      is the duration of delay before the population grow exponentially. \\
      \\
      Before model fitting, I transform biomass and population data into log scale and then fit experiments data to models. 
      For the simple linear phenomenological model, cubic polynomial model, I use R to fit linear model directly. 
      For Logistic and Gompertz nonlinear models, I need to calculate and choose start value of parameters. 
      $N_{0}$ start value is the minimum of biomass or population. $N_{max}$ and K are same, whose start value is 
      the maximum of biomass or population. Start value of r is calculated by linear model, choosing the middle 70 
      percents of biomass or population value range, drawing a straight line and setting the value of slope as start 
      value. $t_{lag}$ starting value is set as the time of the largest second order derivative of population growth 
      point. Furthermore, I sample start values 400 times to find out the best start value combination.\\
      \\
      After model fitting, I calculate AIC, BIC and $R^2$ of these model fittings. I also calculate AICc, 
      which is the second order derivative of AIC and correct small sample size. When size/40 $<$ number 
      of parameters and size $>$ 5, I calculate AICc as a supplementary criteria of model selection. Here is the 
      equation of AICc (k is the number of free parameters and n is the sample size):
      \begin{equation}
        \ AIC_{c} = AIC + \frac{2k^{2} + 2k}{n - k - 1}\
      \end{equation}

    \subsubsection{Plotting and analysis}
      I draw line charts of AIC, BIC and $R^2$ which give a direct discription on 3 models' fitting performance 
      in bacterial growth data. Moreover, the smaller AIC and BIC are and the bigger $R^2$ is, the better 
      model performs. Therefore I calculate the best performing model propotion in AIC, BIC and $R^2$ criteria 
      and visualize them as pie charts. Although only part of datasets meet the condition of using AICc, I analyze 
      it and summarize it with the other 3 model comparison criteria in a table.

  
  \section{Results}
    Finally, 282 of 284 datasets are fitted successfully in all of cubic polynomial model, Logistic model 
    and Gompertz model. Also, all of the model fitting lines are drawn in the point graph of every dataset. 
    Here is an example of the model fitting situation to the last dataset (Figure 1). This dataset contains 
    89 records, which is a relative big sample size among 284 datasets, and the population growth curve is also 
    reasonable. It is obvious that every model shows a good fit to the growth of microbes population, but 
    only Gompertz model catches the lag phase of microbe growth. Therefore, it seems like Gompertz model is 
    the best model to fit this dataset. However, there are some more scientific and reasonable model selection 
    criterias considering fit and complexity and giving a specific number to show model fitting performance, 
    such as AIC, AICc, BIC and $R^2$. Table 1 demonstrates exact number of datasets that every model shows 
    the best performance in AIC, AICc, BIC and $R^2$.
  
    \begin{figure}
      \centering
      \includegraphics[scale=0.5]{../result/modelfitting/284.pdf}
      \caption{An example of 3 models fitting to a dataset}
    \end{figure}

    \begin{table}[htbp]
      \centering
      \caption{The best fitting model under 4 criteria}
      \begin{tabular}{ccccc}
        \hline
        Model & AIC & AICc & BIC & $R^2$ \\ \hline
        Gompertz & 172 & 127 & 169 & 192 \\ \hline
        Logistic & 54 & 112 & 59& 29 \\ \hline
        Cubic & 56 & 25 & 54 & 61 \\ \hline
      \end{tabular}
    \end{table}
  
  \subsection{$R^2$}
    $R^2$ is the residual sum of squares for a model, which is the simplest criterion to show the quality of 
    fit. The more $R^2$ close to 1, the better model fits experiments data. As shown in Figure 2, cubic model 
    shows general large $R^2$, while both of Logistic model and Gompertz model appears to have some quite low 
    $R^2$. Calculated from Table 1, there are $68.1\%$ datasets fitted best by Gompertz model, $21.6\%$ 
    datasets fitted best by cubic polynomial model and $10.3\%$ datasets fitted best by Logistic model. However, 
    maximizing $R^2$ doesn't consider model complexity, neglecting the parsimony principle.
    \begin{figure}
      \centering
      \includegraphics[scale=0.4]{../result/modelfitting/RSq.pdf}
      \caption{Model fitting performance $R^2$}
    \end{figure}

    
  \subsection{AIC \& AICc}
    The Akaike information criterion (AIC) is a model selection criterion that considers both fit and complexity 
    among multiple models. When the sample size is small, AICc should be used to correct the bias. Gompertz model's 
    AIC shows an overall relative low value compared with the other 2 models given Figure 3 line chart. In further 
    best fitting model analysis, Gompertz model, cubic polynomial model and Logistic model fitting real microbes population 
    growth data best accounts for $61\%$, $19.9\%$ and $19.1\%$ of 284 datasets respectively. However, when 
    calculating AICc, there are 18 datasets' size less than or equal to 5, which can not be calculated AICc value. 
    In 264 datasets, best fitting model propotion changes a lot, Gompertz model decreasing to $48.1\%$, cubic 
    polynomial model decreasing to $9.5\%$ and Logistic model increasing to $42.4\%$. But Gompertz model still 
    accounts for the largest propotion of best fitting and shows general lower AICc value in Figure 4.
    \begin{figure}
      \centering
      \includegraphics[scale=0.4]{../result/modelfitting/AIC.pdf}
      \caption{Model fitting performance AIC}
    \end{figure}

    \begin{figure}
      \centering
      \includegraphics[scale=0.4]{../result/modelfitting/AICc.pdf}
      \caption{Model fitting performance AICc}
    \end{figure}

  \subsection{BIC}
    BIC is a model selection criterion taking fit ,complexity and sample size into account. It prefers simpler 
    models, especially sample size increasing. Three models' fitting performance in Bayesian information 
    criterion (Figure 5) is similar with what AIC interprets, Gompertz model showing a globle lower BIC. The best 
    performing model propotion in BIC is $59.9\%$ for Gompertz model, $19.1\%$ for cubic polynomial model 
    and $20.9\%$ for Logistic model.
    \begin{figure}
      \centering
      \includegraphics[scale=0.4]{../result/modelfitting/BIC.pdf}
      \caption{Model fitting performance BIC}
    \end{figure}


  \section{Discussion}
  Population growth is a complex system affected by massive factors. What is a good method to deal 
  with such a complicated system is setting up mathematical models. 3 criteria of a mathematical 
  model is generality, realism and precision \citep{RICHARDLEVINS1966TSOM}. When establishing 
  models, we should balance these 3 criteria according to our research objective. Strictly 
  speaking, three models in this program are all empirical models instead of totally mechanistic 
  models. The purpose of these models is to describe and quantify experimentally observed 
  patterns, but not explain why this particular pattern emerges \citep{Peleg2011}. Therefore, 
  the mathematical convenience and good fit performance matter. Here I use AIC, BIC and $R^2$ 
  to show statistical measure of model fitting, and try to answer whether time lag is worthy 
  of introduced in microbes population growth model.\\
  \\
  In the data management, I deleted a dataset with too high propotion of negative biomass. Therefore, the other 
  284 datasets are fitted by cubic polynomial model, Logistic model and Gompertz model after deleted negative 
  biomass and time. Ultimately, 282 datasets are successfully fitted in all of these 3 models, and the 
  fitting lines seem rational and well-fitted to describe microbes population growth. $R^2$ is used to show 
  the goodness of fit, while AIC, AICc and BIC are used as model selection criteria, considering both fit 
  and model complexity. Cubic polynomial model seems to fit microbes experiments data well with general high 
  $R^2$ value while Gompertz model shows the highest $R^2$ value in 192 datasets. For model selection, 
  Gompertz model shows the lowest AIC and BIC value in majority of 282 datasets, which means although BIC favor 
  simpler model and have correct sample size bias, Gompertz model still performs better than the other 2 models. 
  When considering AICc, it is the same case. In conclusion, Gompertz model is the best of these 3 models to 
  describe microbes population growth.\\
  \\
  The success of Gompertz model demonstrates at least 2 enlightenments. Firstly, mechanistic model is better 
  than totally phenomenological linear model when the number of parameters are the same. 
  Time lag is a worthwhile parameter to be added into microbes population growth data which can improve 
  model fitting performance significantly. The biological reason of this phenomenon may be the transition 
  behavior of microbes at the end of lag phase of population growth process \citep{VerhulstA.J2011Aotl}. 
  Although the value of parameter time lag might not be equal to the biological meaning of lag phase duration.\\
  \\
  Besides, there are many other population growth model not involved in this program. Such as Baranyi 
  model \citep{BARANYIJ1993Ande}, three-phase model \citep{BuchananR.L1997Wisg}, 
  totally and truly mechanistic model on cells level \citep{Peleg2011} and so on. Baranyi model is also 
  a widely used model to describe population growth. However, instead of adding time lag parameter, it added 
  the initial physiological state of cells $h_{0}$ compared with Gompertz model. Furthermore, this model used 
  an equation $h_{0} = t_{lag} r_{max}$, appearing in the solution of rate equation, to calculate time lag. 
  The ralative high correlation between $r_{max}$ and $h_{0}$ results in difficulty of estimating the parameters 
  value of Baranyi model \citep{GrijspeerdtK1999Etpo}. But a more district experiments design can guarantee 
  the precise of parameter estimate. Also the benefits of introducing $h_{0}$ in model should 
  be calculated in the future.\\
  \\
  Whitting specified models into 3 levels \citep{Whiting}. All models mentioned above are all level one model 
  that only describe changes of microbial numbers versus time. Level 2 models demonstrates effect of environment 
  on parameters in level 1. What level 3 models do is combining level 1 and level 2 models and calculating 
  microbial behavior under changed environment condition. We should choose suitable models according to our 
  objective. Such as cubic polynomial model can catch death phase, that may be the reason why in some datasets 
  cubic model shows the best fit. If we need to predict different temperature population growth (level 3 models), 
  Gompertz model has some shortcomes and Baranyi model is the better choice of level 1 model \citep{Peleg2011,SILVA2018}. 
  When using models, we ought to keep in mind that an appropriate model to the question is more important 
  than model fitting performance. The validation of a model is that it generates good testable hypotheses 
  relevent to crucial problems.\\
  \\
  In summary, Gompertz model fits microbes population growth expirical datasets best because of catching lag 
  phase of microbes growth which is consistent with microbes transition behavior observed in experiments. 
  In addition, time lag is worthy to be added into fitting model after balancing fit goodness and model complexity.

  
  \bibliographystyle{agsm}

  \bibliography{model_fitting}

\end{document}